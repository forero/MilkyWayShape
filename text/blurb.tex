\documentclass{article}
\usepackage{natbib}
\newcommand{\apj}{ApJ}  
\newcommand{\apjs}{ApJS}  
\newcommand{\apjl}{ApJL}  
\newcommand{\aj}{AJ}  
\newcommand{\mnras}{MNRAS}  
\newcommand{\mnrassub}{MNRAS accepted}  
\newcommand{\aap}{A\&A}  
\newcommand{\aaps}{A\&AS}  
\newcommand{\araa}{ARA\&A}  
\newcommand{\nat}{Nature}  
\newcommand{\physrep}{PhR}
\newcommand{\pasp}{PASP}    
\newcommand{\pasj}{PASJ}    
\newcommand{\hMsun}{{\ifmmode{h^{-1}{\rm {M_{\odot}}}}\else{$h^{-1}{\rm{M_{\odot}}}$}\fi}}  
\newcommand{\Msun}{{\ifmmode{{\rm {M_{\odot}}}}\else{${\rm{M_{\odot}}}$}\fi}} 
\newcommand{\hMpc}{{\ifmmode{h^{-1}{\rm Mpc}}\else{$h^{-1}$Mpc }\fi}}  
\newcommand{\hkpc}{{\ifmmode{h^{-1}{\rm kpc}}\else{$h^{-1}$kpc }\fi}}  
\newcommand{\kms}{\,km~s$^{-1}$}
\title{Shapes of Milky Way Dark Matter Halos}
\author{Jaime E. Forero-Romero\\ Departamento de F\'{i}sica, Universidad de los Andes\\ Cra. 1 No. 18A-10, Edificio Ip, Bogot\'a, Colombia}



\begin{document}
\maketitle



\section{Observations}

In this section we summarize the constraints on the shape of the Milky
Way dark matter halo.

   
\section{Simulation}\label{Simulations}

We use a large N-body simulation dubbed \verb"Bolshoi". The data in
this paper is available to the public throug a
database \footnote{\texttt{http://www.multidark.org/MultiDark/MyDB}}
presented by \cite{Riebe11}. The \verb"Bolshoi" simulation follows 
the non-linear evolution of dark matter density field in a cubic
volume of side $250$\hMpc sampled with $2048^3$ particles. The code
adaptive mesh refinment code ART was used  \citep{Klypin09}. A detailed
description of this simulation can be found in \cite{Bolshoi}.

The cosmological parameters are compatible with the results from the
fifth and seventh year of data from the Wilkinson Microwave Anisotropy
Probe \citep{Komatsu2009,Jarosik2011}, with $\Omega_m=0.27$,
$\Omega_{\Lambda}=0.73$, $n_{s}=0.95$, $h=0.70$ and $\sigma_8=0.82$
for the matter density, dark energy density, slope of the matter
fluctuations, the Hubble constant at $z=0$ in units of 100km s$^{-1}$
Mpc$^{-1}$ and the normalization of the power spectrum. The mass of a
simulation particle is $m_p = 1.4\times 10^{8}$\hMsun. 


\subsection{Halo finding}

We use halos that were defined using the Bound Density Maxima (BDM)
algorithm \citep{KlypinBDM}. The first step in the algorithm is
finding the density at the particles' positions in the simulation
around which spheres of radius $R$ are built to contain a mass
overdensity $M_{\Delta} = \frac{4\pi}{3}\Delta \rho_{\rm
  cr}(z)R_{\Delta}^{3}$, where $\rho_{\rm}$ is the critical density of
the Universe and $\Delta$ is a desired overdensity threshold. We use
the results obtained for $\Delta=200$.


\subsection{Concentration and shape measurements}


\bibliographystyle{abbrvnat}
\bibliography{references}

\end{document}

\subsection{Concentration estimates}

The estimation for the concentration values is done using an
analytical property of the NFW profile (see Sect.\,\ref{SL_Ana}) that relates the circular
velocity at the virial radius:
\begin{equation}
V_{200} = \left(\frac{GM_{200}}{R_{200}}\right)^{1/2},
\end{equation}
with the maximum circular velocity 
\begin{equation}
V_{\rm max}^{2} = {\rm max}\left[\frac{GM(<r)}{r}\right].
\end{equation}


The $V_{\rm max}/V_{200}$ velocity ratio is used to determine the halo
concentration, $c$ (the ratio between $R_{200}$ and the scale radii of
the NFW profile), using the following relation \citep{Bolshoi}:

\begin{equation}
\frac{V_{\rm max}}{V_{200}} = \left(\frac{0.216 c}{c}\right)^{1/2}
\end{equation}

where $f(c)$ is
\begin{equation}
f(x) = \ln(1+c) - \frac{c}{(1+c)}.
\end{equation}

For each BDM overdensity the $V_{\rm max}/V_{200}$ ratio in order to
find the concentration $c$ by solving numerically the previous two
equations.  This method provides a robust estimate of the concentration
compared to a radial fitting to the NFW profile, which is strongly
dependent on the radial range used for the fit
\citep{Bolshoi,Meneghetti13}. Comparison of these two methods to
using halos where the NFW functional fit yield a small systematic
offset of $(5-15)\%$, with the concentration  derived by the velocity
ratio method being higher \citep{Prada12}.  

\subsection{Shape estimates}


\end{document}
