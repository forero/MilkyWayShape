

   
\section{Simulation}\label{Simulations}

We use a large N-body simulation dubbed \verb"Bolshoi". The data in
this paper is available to the public throug a
database \footnote{\texttt{http://www.multidark.org/MultiDark/MyDB}}
presented by \cite{Riebe:2011gp}. The \verb"Bolshoi" simulation follows 
the non-linear evolution of dark matter density field in a cubic
volume of side $250$\hMpc sampled with $2048^3$ particles. The code
adaptive mesh refinment code ART was used  \citep{Klypin09}. A detailed
description of this simulation can be found in \cite{Klypin:2010qw}.

The cosmological parameters are compatible with the results from the
fifth and seventh year of data from the Wilkinson Microwave Anisotropy
Probe \citep{Komatsu2009,Jarosik2011}, with $\Omega_m=0.27$,
$\Omega_{\Lambda}=0.73$, $n_{s}=0.95$, $h=0.70$ and $\sigma_8=0.82$
for the matter density, dark energy density, slope of the matter
fluctuations, the Hubble constant at $z=0$ in units of 100km s$^{-1}$
Mpc$^{-1}$ and the normalization of the power spectrum. The mass of a
simulation particle is $m_p = 1.4\times 10^{8}$\hMsun. 


\subsection{Halo finding}

We use halos that were defined using the Bound Density Maxima (BDM)
algorithm \citep{KlypinBDM}. The first step in the algorithm is
finding the density at the particles' positions in the simulation
around which spheres of radius $R$ are built to contain a mass
overdensity $M_{\Delta} = \frac{4\pi}{3}\Delta \rho_{\rm
  cr}(z)R_{\Delta}^{3}$, where $\rho_{\rm}$ is the critical density of
the Universe and $\Delta$ is a desired overdensity threshold. We use
the results obtained for $\Delta=200$. 

There a few cases where a halo is truncated to have a radius
$R<R_{200}$. This corresponds to halos that are about to merge with
other massive structures. In this case the radius corresponds to the
distance to the surface where the density raises again due to the
proximity to the soon-to-be host halo.

The particles in the halo are also subject to an unbinding process,
whereby the kinetic energy of each particle is compared against the
gravitational potential. Particles that are found to be gravitational
unbound are removed from the halo.

For each halo the circular velocity $V_{\rm circ}=\sqrt{GM(<r)/r}$ is
calculated using  the radial mass distribution $M(<r)$ for all bound
particles. The bins in radius are $\Delta \log r=0.01$. The maximum
$V_{\rm circ}$ is also stored as $V_{\rm max}$.

\subsection{Concentration measurement}


The concentration values reported in the database are computed for an
spherical NFW profile.  

The measurement of the concentration uses a property of halos by which
the ratio $V_{\rm max}/V_{\rm 200}$ where

\begin{equation}
V_{200} = \left(\frac{GM_{200}}{R_{200}}\right)^{1/2},
\end{equation}

is a measurment of the concentration. For the case of the NFW halo
density profile, this ratio follows:

\begin{equation}
\frac{V_{\rm max}}{V_{\rm 200}} = \left(\frac{0.216
  c}{f(c)}\right)^{1/2}, 
\label{eq:ratio}
\end{equation}

where $f(c)$ is
\begin{equation}
f(c) = \ln(1+c) - \frac{c}{(1+c)}.
\label{eq:c_function}
\end{equation}

After calculating the ratio $V_{\rm max}/V_{\rm 200}$ for each halo
the equations \ref{eq:ratio} and \ref{eq:c_function} are solved
numerically. 

The results of this method are computationally robust, in contrast to
more uncertain radial fitting methoads that are strongly dependent on
the range used for the fit. \citep{Klypin:2010qw,Meneghetti2013}. The
comparison of these two methods yield an offset of $<15\%$, where the
concentrations derived with the velocities are
higher. For halos with concentrations $c<5$ the offset is smaller than
the intrinsic scatter at fixed halo mass \citep{Prada2012}.

\subsection{Shape measurement}
The information on the axis ratios is obtained by diagonalizing the
modified inertia tensor computed from all the bound particles inside
the halo radius

\begin{equation}
T_{jk}  = \sum_{i}\frac{x_{ij}x_{ik}}{r_i^2}
\end{equation}

where $i$ runs over the particle index, $j,k=1,2,3$ corresponds to
the dimensions of the position vector and $r_i$ is the norm of the
position vector measured from the halo center. The eigenvalues of this
tensor are proportional to the values of $a$, $b$ and $c$. The ratios
$b/a$ and $c/a$ correspond to the eigenvalue ratios.

The values stored in the database do not include any correction due to
the fact that $T_{jk}$ is calculated on a spherical region. However we
apply such correction, which is dependent on the halo concentration
$q=R_{\rm rms}/R_{\rm vir}$ where $R_{\rm rms}=\sum_i m_iR_i^2/\sum_i
  m_i$. The true axial ration are computed using the following
  formulae \citep{Riebe:2011gp}:

\begin{equation}
\left(\frac{c}{a}\right)_{\rm true}  =
\left(\frac{c}{a}\right)^s,\ s=1 + 2{\rm max}(q - 0.4,0) +(5.5{\rm max}(q-0.4,0))^3
\end{equation} 

\begin{equation}
\left(\frac{b}{a}\right)_{\rm true}  =
\left(\frac{b}{a}\right)^p,\ p=1 + 2{\rm max}(q - 0.4,0) +(5.7{\rm max}(q-0.4,0))^3
\end{equation} 
